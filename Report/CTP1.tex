\documentclass[conference]{IEEEtran}
\IEEEoverridecommandlockouts
% The preceding line is only needed to identify funding in the first footnote. If that is unneeded, please comment it out.
\usepackage{cite}
\usepackage{amsmath,amssymb,amsfonts}
\usepackage{algorithmic}
\usepackage{graphicx}
\graphicspath{ {img/} }
\usepackage{hyperref}
\hypersetup{
    hidelinks
}
\usepackage{textcomp}
\usepackage{xcolor}
\usepackage{wrapfig}
\def\BibTeX{{\rm B\kern-.05em{\sc i\kern-.025em b}\kern-.08em
    T\kern-.1667em\lower.7ex\hbox{E}\kern-.125emX}}
\begin{document}

\title{Computer Technology Project I\\
{\large TurtleBot3 - Robot Design}
\thanks{Identify applicable funding agency here. If none, delete this.}
}

\author{\IEEEauthorblockN{Daniel Pihl}
\IEEEauthorblockA{\textit{Århus University} \\
AU712814}
\and
\IEEEauthorblockN{Steffen Petersen}
\IEEEauthorblockA{\textit{Århus University} \\
AU722120}
}

\maketitle

\begin{abstract}
This project aimed to implement autonomous navigation for a Turtlebot3 robot using the Robot Operating System (ROS), 
with a primary objective of obstacle avoidance and achieving a high average linear velocity in a simulated "search and rescue" 
scenario. The robot navigation was designed based on the "see - think - act" concept, using LIDAR and RGB sensing to 
detect the environment and simulated victims. The existing ROS framework was utilized and extended on a Raspberry Pi 3 
to control the robot and establish communication with the sensors. \\\\
The navigation algorithm underwent several iterations to enhance the robot's movement, taking into account the 
trade-offs between linear and angular speed. The final implementation incorporated techniques for smooth turns, 
corner avoidance, and collision handling. The system's performance was evaluated through experiments conducted in a 
controlled environment featuring a predefined obstacle course.
The results demonstrated successful autonomous navigation, effective obstacle avoidance, and good speed while minimizing 
collisions. This project provided valuable insights into ROS and the principles of robot implementation, sensor integration 
and navigation algorithms.
\end{abstract}

\section{Introduction}\label{intro}
In this project, our main goal was to implement a Turtlebot3 robot for a simulated version of search and rescue purposes.\\
This would involve creating a program to allow the robot to move autonomously and avoid obstacles, utilizing the 
Robot Operating System (ROS), all while searching for targets on the floor.\\
We also aim to optimize this robot for the highest possible speed, through an efficient navigation logic, 
following the "see - think - act" concept.\\

\section{Specifications}\label{specs}
\subsection{TurtleBot3 Burger}
In this project we are using the TurtleBot3 Burger robot, to practice our ROS based robot programming.\\
The TurtleBot3 Burger is equipped with a Raspberry Pi 3 Model B \cite{b1} in combination with an OpenCR 1.0 board, to 
provide us with options for programming and controlling it. \\

\subsection{LDS-01 LIDAR}
Additionally the Burger has an LDS-01\cite{b2} Lidar scanner mounted on top, which we use for the navigation of the Burger.\\

\subsection{Raspberry Pi 3 Model B}
The Raspberry Pi is a small-scale computer ideal for hobbyist purposes, with GPIO pins, that we could use to add extra sensors to
our robot \cite{b3}.\\

\subsection{RGB Sensor}
Using the extra IO support of the Raspberry PI, we could add the ISL-29125 RGB sensor \cite{b4}, that allows us to simulate scanning for
victims.\\

\section{Process}\label{process}
In this section we will elaborate on the process of developing our own understanding of the problem, and the solutions we have 
come up with along the way.\\
\subsection{Design and Implementation}\label{design}
In this section we are going to talk about our own thoughts on design and implementation but in an abstract way.\\
Concrete implementation and thoughts will come in the later sections. \\
\\
In the early stages of our process, we started out with the RGB sensor as it was a component that just had to be mounted 
on the robot and connected to the Pi. If we could make that work, it was a simple matter of just implementing it later.\\
Our intention was to make the sensor read some coloured tags from the floor, which would simulate victims that our robot 
had to recognize as it moves around.\\
Firstly we installed the necessary libraries on the Pi, \textit{smbus} to communicate with the sensor, and \textit{time} for 
the initialization. \\
To start working on the implementation, we were given a template, that already had the code for communicating through 
\textit{smbus} with the sensor, to configure the registers of the sensor. We then had to add to that, a way to extract the red, 
green and blue values from the registers within the sensor, so that we could use these in our to-be robot code, and detect 
victims.\\
After we got the RGB sensor working we moved on to theorize what implementation we wanted to use for navigation with the robot.\\
Our Turtlebot3 was equipped with a LIDAR sensor which is capable of scanning 360° with a laser, that could measure distance. 
It would return data in the form of an array like so:
\[dist = \left[d_0, d_1, ..., d_{359}\right]\]
Where $d_i$ is the distance to nearest object at $i$ degrees from the front of the sensor, going counterclockwise. 
In the first implementation we would simply look at a span of $120^{\circ}$ that we would divide into three subcategories:
\[left = \left[d_{60}, d_{59}, ..., d_{15}\right], N = 45\]
\[front = \left[d_{14}, d_{13}, ..., d_0, d_{359}, d_{358}, ..., d_{345}\right], N = 30\]
\[right = \left[d_{344}, d_{343}, ..., d_{300}\right], N = 45\]

The reason we chose to only look forwards, was due to us thinking it would be smarter for the robot
to simply turn instead of having to go backwards, since this would allow the robot to achieve an overall 
higher average linear speed. \\
Using these cones, we would have to specify what the robot would do in different cases, 
which we implemented with if/else statements. Each of the abstract cases can be seen in the illustrations below. 
Note that in Case 1, we would ideally make a sharp swing turn, instead of reversing out of the corner.
\begin{figure}[h] %this figure will be at the left
    \centering
    \includegraphics[width=0.5\textwidth]{cases.png}
    \caption{Theoretical cases for navigation}
\end{figure} 

Moving on, we could now get to work practically with the real robot, and at first we would need to set up ROS and 
get familiar with it, before running our own code on it. Up until now it had all been mostly theoretical, 
which would only get us so far.\\
ROS is a necessary tool to help us control the robot. In short it is a collection 
of tools, libraries and conventions that allows us to simplify the creation of complex and robust robot behavior.\\
Having ROS installed, we needed to get to know it, and to that end we followed a tutorial. We downloaded the 
Turtlebot3 packages to set up a virtual environment using ROS features, where we would be able to drive a simulation 
of a turtle around a little screen with our keyboard. \\

From here on and till the end of our timeframe, we continuously worked on the Robots navigation and optimization.
We were presented with a sort of starter build of the code \cite{b5} for our robot's obstacle avoidance.\\
The task at hand first was to update the given code and make the robot read from a different 
span of angles using the LIDAR. \\
\vspace*{2pt}\\
At first we wanted to implement a way to read from angles -45° to 45° to 
represent left and right. The current aim was to design according to the following picture\cite{b6}
\begin{figure}[h] %this figure will be at the left
    \centering
    \includegraphics[width=0.5\textwidth]{initialNav.png}
    \caption{Simple navigation logic}
\end{figure} \\
We got the simple navigation to work and wanted to implement a reading from both Lidar and the RGB sensor 
in a frequency that made sense. After we got the basics down, had a robot that could do simple maneuvering 
and send relevant data from both LIDAR and RGB sensor, it was time to optimize the navigation further.\\
We would start by optimizing how the robot navigates and move on to optimizing the linear speed, where we had 
to consider the linear speed vs the angular speed. In short, if we were to move faster forwards, we wouldn't be 
able to make a sharp turn due to the higher speed. Likewise, if we must make a sharp turn, we won't be able to have 
very high speed.\\
To achieve a higher linear speed whilst turning would want to break the turn into smaller steps. 
If our robot would be facing an obstacle, we want to calculate how much it should turn. Instead of making 
the entire turn in one go, we wanted to apply half of the angle and store how much angle there is left 
to apply it at the next publishing to complete the entire turn. See the following two pictures. \cite{b7}\\
\begin{figure}[h] %this figure will be at the left
    \centering
    \includegraphics[width=0.4\textwidth]{betterNav1.png}
\end{figure}\\
\begin{figure}[h] %this figure will be at the left
    \centering
    \includegraphics[width=0.4\textwidth]{betterNav2.png}
    \caption{Speed Optimization for turns}
\end{figure} \\
By doing this, we would in theory be able to always achieve a higher linear velocity whilst still avoiding 
obstacles in our way.\\
The final two things we wanted to implement was a way to calculate the average linear speed and to optimize 
that so we would have the highest possible linear speed over the course of the time our robot was driving, 
and a collision counter, which would increment if the distance to an obstacle was lower or equal to 4cm for 
the front of the robot or 5cm for other angles.\\
Our goal was to keep the linear speed as high as possible and the collision counter as low as possible.\\

\subsection{Experiment setup and Results}\label{experiment}
\subsubsection{Method}\label{method}
During our process of making the program for the robot, we have been using a reiterated model called the waterfall model.
This model is perfect for the structure of our project. Firstly, we would analyze the requirements. \\
Then we make decisions based on how we think the robot should operate in different situations (System Design).\\
Then we would make an implementation of our code (Implementation), and lastly, test the implementation by 
letting the robot run, and observing it (Testing). Whatever failures we might find with the program, 
we could then reiterate the process by either going back to system design, if our design choices were wrong, 
or go to change the implementation if it simply behaved unexpectedly. See the following illustration.\\
\begin{figure}[h] %this figure will be at the left
    \centering
    \includegraphics[width=0.4\textwidth]{waterfallmodel.png}
    \caption{Waterfall Model (selfmade illu.)}
\end{figure} 
\\
\subsubsection{ROS Setup}\label{ros}
As previously mentioned, before we could begin truly working with the actual robot, we needed some abstraction in place, 
in this case ROS.\\
ROS allows us to use packages that have previously been used for working with this robot platform, 
and then quite simply customize our own node that can run our implementations of code. \\
To familiarize ourselves with the workings of ROS, we did start by running through the tutorial of 
virtually running a simulation using the different features of ROS. Learning the uses for ROS topics, packages, 
nodes and messages.\\
\\
In practice, we used the turtlebot3\_example and \\turtlebot3\_bringup packages from the turtlebot3 repository \cite{b8}. \\
The bringup package contained a launch file that ensured the necessities for the turtlebot3 would be running, 
starting the LIDAR and ROScore. On top of this, we could use the turtlebot3\_example's turtlebot3\_obstacle node 
to serve as a base for our own implementation of obstacle avoidance and navigation.\\
\\
This was handy as it already had the code for setting up the Twist message from geometry\_msgs and a topic to allow 
us to control the motors on the robot simply using this premade message.\\
Additionally it had an example of how to read the data from our LIDAR, also setting up a listener for the scan message, 
from the LaserScan package.\\
That was about the end of what we kept from the example however, as the actual implementation of the \textit{obstacle()} function 
is completely changed, both in terms of principle and code. Additionally we did modify the \textit{get\_scan()} function, 
though it's principally similar. The main difference, is that we changed how the filter worked, as invalid readings 
in our implementation should not be set to 0.
\\\\
\subsubsection{Test setup}\label{test}
Following the waterfall model, we would need to test each implementation of code, to determine what changes should be made.\\
We set up a little obstacle course made from cardboard boxes to function as walls, and some red paper dots on the ground 
to simulate victims to be found. Our idea was the robot should navigate collision free whilst moving around without either 
getting stuck in a loop where it would go in circles, or just getting stuck overall. \\
The robot initially had a few problems with the course. The first struggle was it would register itself too close to an 
obstacle and immediately try to turn, but as it turned a new obstacle would be spotted and it would only move a little 
bit before turning again, creating a zig-zag effect. \\Another problem arose when we tried fixing the first problem, that 
it would simply get too close to an obstacle and then get stuck on a wall because we turned too late. \\
Our optimization and final solutions follow in the next section.
\\\\
\subsubsection{Optimization}\label{opti}
In our first system design, the idea was that we wanted to maintain the linear speed constant throughout, and simply 
decide to turn more or less depending on how close we were to an obstacle. \\
We attempted this by partitioning the LIDARs view, as we have illustrated below, into a front cone and 2 cones on both 
the left and right side of the bot.\\
In the illustration (Fig. 5) we have also included some red blobs to explain a scenario where these are obstacles. \\
We would only decide to turn if something was in the front cone, as we would otherwise be able to continue. 
When an obstacle was found in front, we would evaluate if there was more room on the left or right side of the bot, 
to decide which way we should turn. In this case we would turn right, as the left has the closer obstacle. 
To determine how much we would turn, we then had different cases, where we first determine if “Left 1” has an 
object within a certain distance, and if it didn't we would check “Left 2”. The amount to turn was then determined 
by a constant (different from “Left 1” to “Left 2”) plus a variable fraction, based on the distance to the obstacle. 
This allowed the amount we turned to be relative to how close an obstacle was, which should in theory give us smoother turns. \\
\begin{figure}[h] %this figure will be at the left
    \centering
    \includegraphics[width=0.5\textwidth]{turtleExample.png}
    \caption{Design showcase and Example}
\end{figure} \\
This implementation however, took many iterations and wasn't quite working, we found the bot would be too stubborn with 
going forward, and unable to avoid obstacles if it ever got into a corner that was too tight. 
This would likely be because the decision for turning would be too narrowly minded, as we don't have a 
full mapping of the terrain, just deciding based on which side had the closer wall. 
This would eventually get it to turn into corners, but turn too little to avoid a wall on the right, 
when the left wall was just a little closer. We needed a new edge case to deal with being cornered.
\\\\
During the later iterations, we noticed some strange decisions from the bot, and decided to add some debugging print
statements, to allow us to see which cones saw what. At this moment we realized we had in fact messed up the cones 
relative to the LIDAR, as we had reversed it entirely, assuming that the array of data from the LIDAR would be ordered 
chronologically with the direction it spins. \\
However this turned out to not be the case, and it was in fact reversed. 
Still we decided to rework the principles on which it should turn slightly. \\
Moving to the next iteration, we would instead only evaluate ±90° from the front, 
still keeping the front cone, but only using one left and right cone, 
and still maintaining the fraction based turn decisions. \\
Of course here we would also add solutions to combat getting stuck in a corner, even if this meant slowing down or backing up slightly. 
This was done by adding a new check, to determine if there was an obstacle very close to an even narrower front cone, at just ±15°, 
this would usually only be triggered if a previous turn had put our nose close to another wall. 
In this case we would stop moving forward, and simply continue turning until our entire front-cone 
was cleared of any immediate obstacles. This of course sacrifices a good bit 
of the average linear speed, however it felt like it was the better option compared to getting stuck. 
Additionally we did add another measure if we ended up being truly stuck, by counting the number of repeated turn decisions, 
if we repeated the same loop too often, we decided it likely meant we were stuck on a corner, as had been observed during testing, 
and then we could reverse out of this situation, to make a new decision instead.
\\\\
\subsubsection{Program design}\label{program}
Our program has its roots in the Turtlebot3\_example package \cite{b5}, using a similar structure for the \textit{Obstacle} class,
additionally we added the simple RGB Sensor class from our previous work, to initialize the sensor for use in the main loop.
\\\\
Below we will describe the details of our obstacle avoidance loop, and what each iteration goes through.\\
The entire code can be found in our GitHub repository \cite{b9}, in "Code samples/FinalNavigation.py".
\begin{itemize}
\item \textbf{Get LIDAR data}\\
Starting off each loop, we must get a new reading from the LIDAR to be able to navigate with current positional data.\\
For this we have the \textit{get\_scan()} function, which is quite similar in structure to the one from the template.
\begin{figure}[h] %this figure will be at the left
    \centering
    \includegraphics[width=0.5\textwidth]{get_scan.png}
    \caption{\textit{get\_scan()} function}
\end{figure} \\
We listen for a new message from the LIDAR and then proceed to filter this data sorting out 
values like infinity, 0 and NaN, setting these entries to 10 instead, which we know is not going to affect our navigation.\\
\item \textbf{Sort and Evaluate Data}\\
Having received the filtered data from the \textit{get\_scan()} function, we need to organize it and evaluate it,
to be able to use it for navigation.
\begin{figure}[h] %this figure will be at the left
    \centering
    \includegraphics[width=0.5\textwidth]{evaluateAndSort.png}
    \caption{Organize data}
\end{figure}\\
We do this by creating a new array, intended to store our readings in order from left to right, starting at -90 degrees up 
to 90 degrees. Since our LIDAR returns an array of 360 readings, one from each degree of angle, we take the first 90 
readings which are on the left side of the bot, and reverse this, because it by default is stored right to left.
And we then add to that, the final 270th reading up to the 360th reading, which would be on the right side of the bot, also 
reversing this.\\
After this we simply use the new array, to create cones for viewing zones of the bot, and then proceeed to store the readings of
shortest distance in each of these relevant areas, to use for evaluation in our navigation.\\
\item \textbf{Check for Victims}\\
In order to check for victims, we read the current red and blue values from the RGB sensor, as the tags we are looking for a red,
we know to look for a higher value of red and lower value of blue. These new values are compared to an older baseline value, 
to avoid reading double. The first baseline is made in initialization of the obstacle avoidance loop.
\begin{figure}[h] 
    \centering
    \includegraphics[width=0.5\textwidth]{victimCheck.png}
    \caption{Victim searching code}
\end{figure} \\
We've decided that 20\% was a good amount to allow the readings to fluctuate, and whenever it has changed more than this,
it's because we've either travelled over a tag we need to read, or we've just come off a tag we've already read. This means
we can create a new baseline here, to compare future readings with. When this happens, we need to also check if we did come across
a new tag, or if we went off of one, so this is where we use the new readings from the sensor, to determine if we are over a red 
tag, and if so increment our counter.\\
\item \textbf{Check for collision}\\
Checking for collisions we need to evaluate on the LIDAR readings, we've decided to simplify the check quite a lot, and
so we simply check if the robot is within 5cm + the LIDAR error. 
\begin{figure}[h] 
    \centering
    \includegraphics[width=0.5\textwidth]{collisionCheck.png}
    \caption{Collision counter code}
\end{figure}
\vspace*{-0.2cm}\\
To avoid any double counting, we have implemented a cooldown 
counter, that decreases with every loop-cycle, so we shouldn't read multiple collisions in the same corner, 
as we would be gone by then. \\
\item \textbf{Navigation control}\\
The navigation controls consists of a few steps, firstly we need to check for a flag to see if we have been marked as 
cornered from a previous loop, and if so, we must continue to turn until our front-cone is clear of immediate obstacles.\\
If we haven't been marked as stuck, we can move on to see, if there are obstacles in front of us, within our set FRONT\_SAFE\_DIST,
and if we are clear, we jump to the else case, which simply lets us continue forward at full speed. \\
Should we instead have obstacles, we need to turn to avoid them, so we immediately decide which way we need to turn, comparing
the evaluation variable for the left side, to the right side, and storing the decision in another flag called rightTurn.\\
Moving on, we check our narrow front cone, for any very close obstacles, this is to make sure we have room to turn at all, and if
we don't, we set the cornered flag and proceed to back up slightly, and the next n-loops would then keep us turning until we are
clear to move forward. \\
If all instead goes well, and our narrow front is clear, we either go to the left or right turn case depending on the flag, 
and here we firstly have a default case that updates the Twist message variables through \\\textit{driveUpdate(angular, linear)},
to a small turn. \\
We then move on to check if the obstacle is within our set SAFE\_STOP\_DISTANCE, to determine if we should turn more than the
default. If we should, we set a new driveUpdate, with a variable amount of turning, depending on the distance to the nearest obstacle.\\
The final detail of the control loop, is that we have added counters to check for repeated entries into the turning loops. This is
to combat getting stuck on corners, so if we enter the same loop to many times in a row, we are most likely stuck, and so we 
reverse instead of continuing to turn into the wall.\\
Once all of the above has been executed, we have the correct Twist message values as is updated by the \textit{driveUpdate(a, l)}
function, and so we can publish these to ROS and have the robot follow our control logic.
\\\\
\begin{figure}[h] 
    \centering
    \includegraphics[width=0.5\textwidth]{Navigation.png}
    \caption{Navigation code}
\end{figure} 
\pagebreak
\item \textbf{Update variables}\\
To finalize the loop, we just need to keep track of the average speed of the robot, this is done using the linear speed we've
saved to the twist.message and a simply counter for how many loops we've been through.
\begin{figure}[h] 
    \centering
    \includegraphics[width=0.5\textwidth]{averageSpeed.png}
    \caption{Continous update of average speed}
\end{figure} 
\end{itemize}


\section{Discussion}\label{discussion}
The thing that took a big toll on the progress of our development was the LIDAR sensor. 
In this section our main topic will be the problems regarding the sensor and how we tried to fix it and work around it. 
As mentioned earlier we encountered a lot of problems with the LIDAR sensor. 
One of them came out to be our understanding of how it worked wasn't correct. When addressed the problem resolved itself 
rather quickly. \\
However, the sensor wasn't without fault either. A lot of the readings we receive from it would be faulty. 
In those situations, our \textit{get\_scan()} method would make that reading a high number so it would be evaluated to 
something far away. We tried setting the value to zero, but this resulted in our cones readings would get skewed and 
make the robot turn and stop a lot more than what was acceptable. The main problem was that these faulty readings seemed 
almost random, and could cause navigation issues that weren't fixable.

Our initial goal of achieving a robot that would move autonomously, avoid obstacles, and detect 
victims was an overall success. \\
We managed to successfully implement our initial and later improved theories. 
With that in mind it is likely possible to achieve better navigation than what we've accomplished. 
One way to make some improvements would be to make the robot more “aware” of how big it is. We could make 
the robot evaluate in a radius around itself 360 degrees, instead of our front-focused approach. 
The radius would have to be a bit bigger than the actual radius to mitigate potential problems when turning around. 
This would then be its own function that would run before anything else in the control loop. 
If a reading from this is returned and is within a specified safety range, it would then turn and make the 
front as far away as possible from the found obstacle. \\\\
A way to improve the navigation further could also have been to map the surroundings using something like 
SLAM (Simultaneous Localization and Mapping), which is a method used to map and localize our robot in the surroundings. 
This way we could let the robot map the unknown environments and thus let the robot make informed decisions based on the map. 
This might however be a bit out of the scope of the project presented to us, but none the less be a way to improve navigation. 



\section{Conclusion}\label{conclusion}
Even though we've been working with very simple programming and robots, the principles of working on a project 
that is a bit bigger in size, requires a lot of time investment. 
We've learned the foundation of understanding a bit more complex robot behavior, and the importance of 
being structured and working systematically instead of being all over the place and hoping for the best. \\
ROS provides a structure and playground for working on these complex projects and has improved our capabilities 
to work on further projects which might be even more complex than this. \\\\
We've presented a way to implement basic autonomous navigation which resulted in reasonably good obstacle avoidance 
and average speed. It works best in more sporadic and spread-out environment in terms of obstacles, 
whereas in tight spaces it could face problems where it couldn't determine whether there was enough space. \\\\
The LIDAR sensor presented us with some issues and would get bad readings that had to be mitigated. 
We've managed to work around this as best as possible, and the navigation functions consistently despite this. \\
We've also discussed ideas to improve the navigation, using different principles than our implementation. 
One way to do it, would be to utilize the full 360 degrees of scanning, and accept closer calls with obstacles, 
and less smooth turns. Another more advanced way, is to continuously map the environment, this should also eliminate 
the significance of bad readings. \\\\
Overall, our navigation system implementation was rather effective.
We maintained a high linear velocity whilst still avoiding different obstacles in a smooth way. \\

\section{Personal Contributions}
During the entire process we discussed internally how we were supposed to divide the many tasks between us.\\
We didn't want to divide us, so we did a lot together in intense collaboration. One might have written some 
code from home before we met and then we would discuss it thoroughly before we just implemented it on the robot. 
Or one might have had an idea of how to calculate a certain thing, and then we would discuss it. 
When working with the robot and the VM we mainly used Steffens computer due to Daniels having some problems 
setting it up and it cost a lot of time we didn't have. \\\\
In the late stages, Daniel focused on writing the report and keeping a birds eye view of the structure, 
while Steffen kept on improving the robot and fixing the problems that we've encountered.\\\\
Regarding the paper, Steffens focal point has been the following:
\begin{itemize}
\item Section \ref{specs} Specifications
\item Section \ref{opti} Optimization
\item Section \ref{program} Program design
\end{itemize}
While Daniel was focused on the following:
\begin{itemize}
\item Section \ref{design} Design an Implementation
\item Section \ref{method} Method
\item Section \ref{ros} ROS setup
\item Section \ref{test} Test setup
\item Section \ref{discussion} Discussion
\item Section \ref{conclusion} Conclusion
\end{itemize}

\begin{thebibliography}{00}
\bibitem{b1} \raggedright Robotis e-Manual - TurtleBot3 Specifications, visited 03-05-2023, \url{https://emanual.robotis.com/docs/en/platform/turtlebot3/features/#specifications}.
\bibitem{b2} \raggedright Robotis e-Manual - LDS-01 Specifications, visisted 03-05-2023, \url{https://emanual.robotis.com/docs/en/platform/turtlebot3/appendix_lds_01/}
\bibitem{b3} \raggedright Raspberry Pi 3 Model B Documentation, visisted 03-05-2023, \url{https://www.raspberrypi.com/products/raspberry-pi-3-model-b/}
\bibitem{b4} \raggedright ISL-29125 RGB Sensor Datasheet, visisted 28-05-2023, \url{https://cdn.sparkfun.com/datasheets/Sensors/LightImaging/isl29125.pdf}
\bibitem{b5} \raggedright Turtlebot3 Package, template code, visisted 28-05-2023, \url{https://github.com/ROBOTIS-GIT/turtlebot3/blob/master/turtlebot3_example/nodes/turtlebot3_obstacle}
\bibitem{b6} \raggedright BrightSpace Week 7, Navigation Slides
\bibitem{b7} \raggedright Brightspace week 9, Speed Optimization Slides
\bibitem{b8} \raggedright Turtlebot3 master repository, visisted 28-05-2023, \url{https://github.com/ROBOTIS-GIT/turtlebot3/tree/master}
\bibitem{b9} \raggedright Our GitHub repository, visisted 29-05-2023, \url{https://github.com/Dan-jpg2/CTP-1}

\end{thebibliography}

\end{document}
