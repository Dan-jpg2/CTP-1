\documentclass{article}
\usepackage[utf8]{inputenc}
\usepackage{graphicx}
\usepackage{geometry}
\geometry{a4paper, total={16cm, 24cm}, top=2cm}
\usepackage{amsmath}
\usepackage{amssymb}
\usepackage{blindtext}
\usepackage{hyperref}
\usepackage{wallpaper}
\usepackage{siunitx}
\graphicspath{ {img/} }
\title{\Huge Digital Circuits\\\LARGE Lab Exercise 1} % OLD SHIT
\author{Steffen Petersen | au722120 \\ Daniel Pihl | au712814} % OLD SHIT
\date{14. October 2022} % OLD SHIT
\hypersetup{
    colorlinks=true,
    linkcolor=black,
    filecolor=magenta,      
    urlcolor=cyan,
    pdfpagemode=FullScreen,
}
\setcounter{tocdepth}{4}
\setcounter{secnumdepth}{4}

\begin{document}
\ThisULCornerWallPaper{1}{background1.png}

\begin{titlepage} % Suppresses displaying the page number on the title page and the subsequent page counts as page 1
  \newcommand{\HRule}{\rule{\linewidth}{0.5mm}} % Defines a new command for horizontal lines, change thickness here

    \center % Centre everything on the page
    
    %------------------------------------------------
    %	Headings
    %------------------------------------------------
    \textbf{\space}\\[2cm]   
    \textsc{\LARGE Århus University}\\[1.5cm] % Main heading such as the name of your university/college
    
    \textsc{\Large Computer Technology}\\[0.5cm] % Major heading such as course name
    
    \textsc{\large Project 1}\\[0.5cm] % Minor heading such as course title
    
    %------------------------------------------------
    %	Title
    %------------------------------------------------
    
    \HRule\\[0cm]
    
    {\huge\bfseries Robot Design\\[0.1cm]}
    {\large\bfseries Turtlebot3\\}
    
    \HRule\\[1.5cm]
    
    %------------------------------------------------
    %	Author(s)
    %------------------------------------------------
    
    \begin{minipage}{0.4\textwidth}
      \begin{flushleft}
        \large
        \textit{Authors}\\
        Steffen T. Petersen | AU722120\\
        Daniel Pihl | AU712814
      \end{flushleft}
    \end{minipage}
    \begin{minipage}{0.4\textwidth}
      \begin{flushright}
        \large\vspace{-4mm}
        \textit{Instructors}\\
        Jalil Boudjadar\\
        Mirgita Frasheri
      \end{flushright}
    \end{minipage}
    
    % If you don't want a supervisor, uncomment the two lines below and comment the code above
    %{\large\textit{Author}}\\
    %John \textsc{Smith} % Your name
    
    %------------------------------------------------
    %	Date
    %------------------------------------------------
    
    \vfill\vfill\vfill % Position the date 3/4 down the remaining page
    
    {\large May XXth, 2023} % Date, change the \today to a set date if you want to be precise
    
    %------------------------------------------------
    %	Logo
    %------------------------------------------------
    
    %\vfill\vfill
    %\includegraphics[width=0.2\textwidth]{placeholder.jpg}\\[1cm] % Include a department/university logo - this will require the graphicx package
     
    %----------------------------------------------------------------------------------------
    
    \vfill % Push the date up 1/4 of the remaining page
    
  \end{titlepage}

%\maketitle
%\includegraphics[width=\linewidth, keepaspectratio=true]{tempFront}

\pagebreak

\section*{Abstract}
\addcontentsline{toc}{section}{Abstract}
Define what have we done and talked about in the report.\\
Set up the general "question" for the report to answer\\


\vspace{11cm}
\tableofcontents
\pagebreak
%\section*{Main body}
\addcontentsline{toc}{section}{Specifications} \addtocounter{section}{1}
\section*{Specifications}
\subsection*{Boolean Algebra}\addcontentsline{toc}{subsection}{Boolean Algebra}
What are the Specifications of the project, what was the purpose here?\\
What do we have to work with?\\
Details about our robot, and it's sensors system etc.\\


\subsection*{Karnaugh Maps}\addcontentsline{toc}{subsection}{Karnaugh Maps}
Karnaugh maps or K-maps, is a way to simplify boolean expressions which are too tedious for Boolean algebra. The reduction could be
done with Boolean algebra. However, with the Karnaugh map it is faster and easier.

\pagebreak

\addcontentsline{toc}{section}{Process} \addtocounter{section}{1}
\section*{Process}
\noindent In the following three subtasks we are to simplify the given expressions using boolean algebra.
\subsubsection*{a.} \addcontentsline{toc}{subsection}{a.}
\[Y1= A\bar{B} + A(\overline{B+C}) + B(\overline{B+C})\]
Applying DeMorgan Theorem we get:



\pagebreak


%\addcontentsline{toc}{subsection}{Testing and Results} \addtocounter{subsection}{1}
%\subsection*{Testing and Results}
\section*{Discussion}
\addcontentsline{toc}{section}{Discussion} \addtocounter{section}{1}
How did things go in this project? \\
Did we do what we wanted, is the robot working, did we learn anything?\\

\section*{Conclusion}
\addcontentsline{toc}{section}{Conclusion} \addtocounter{section}{1}
Wrap up the abstraction, is it achieved, was the project a success?

\vspace{3mm}




\end{document}